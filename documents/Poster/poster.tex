\documentclass[final]{beamer}
% beamer 3.10: do NOT use option hyperref={pdfpagelabels=false} !
% \documentclass[final,hyperref={pdfpagelabels=false}]{beamer} 
% beamer 3.07: get rid of beamer warnings

\mode<presentation> {  
%% check http://www-i6.informatik.rwth-aachen.de/~dreuw/latexbeamerposter.php for examples
  \usetheme{Durham} %% This points to the theme cooked up by the final year tutor
}


\usepackage[english]{babel} 
\usepackage[latin1]{inputenc}
\usepackage{amsmath,amsthm, amssymb, latexsym}

  \usefonttheme[onlymath]{serif}
  \boldmath
  \usepackage[orientation=portrait,size=a0,scale=1.4,debug]{beamerposter}                       

  % e.g. for DIN-A0 poster
  % \usepackage[orientation=portrait,size=a1,scale=1.4,grid,debug]{beamerposter}
  % e.g. for DIN-A1 poster, with optional grid and debug output
  % \usepackage[size=custom,width=200,height=120,scale=2,debug]{beamerposter} % e.g. for custom size poster
  % \usepackage[orientation=portrait,size=a0,scale=1.0,printer=rwth-glossy-uv.df]{beamerposter}
  % e.g. for DIN-A0 poster with rwth-glossy-uv printer check ...
  %

  \title[Thermal Re-ID]{Camera-to-Camera Tracking for Person Re-identification within Thermal Imagery}
  \author[G Ingram]{Thomas Robson - Supervised by Dr Toby Breckon}
  \institute[Durham]{School of Computer Science, Durham University}
  \date{16th April 2012}

  \begin{document}
  \begin{frame}{} 

  \vfill
  \begin{block}{Introduction}
          A fundamental task for a distributed multi-camera surveillance system is Person Re-Identification, or to associate people across different camera views. This has been well researched in colour, but there has been very little research done on solving this problem in thermal, making our work state of the art. 

The intent of this project is answer the question "Which features of a human target are appropriate to facilitate Re-Identification in thermal video?" and to develop a functioning Re-Identification system. The challenges associated with this are due to the increased complexity of thermal features compared to colour features.
        \end{block}
        
    \begin{columns}[t]
      \begin{column}{.49\linewidth}
        
        \begin{block}{Person Detection and Tracking}
        	The process of Person detection and Tracking employed in this Project can be broken down into multiple stages. 
        	\begin{itemize}
        	\item Background Subtraction. This is done using the Mixture of Gaussians (MOG) method, allowing the system to learn a background model and comparing each new frame to this. 

        	\item Person Identification. This is done by performing contour dectection on a foreground target and using the Histogram of Oriented Gradients (HOG) to determine whether this target is a person.
        	
        	\item TLD Tracker
          \end{itemize}
        \end{block}
       


        \begin{block}{Network Architecture}
		explain arch
		add in arch diagram
          
        \end{block}
	
        \begin{block}{Network Results 1}
		training graphs
		classification report and confusion matrices

        \end{block}
	 \end{column}
	 \begin{column}{.49\linewidth}
 		\begin{block}{Network Results 2}
 		euclidean distance plots
 		
 		%\includegraphics[width=.90\linewidth]{../combo.png}  
         
        \end{block}
        
        \begin{block}{The Re-Identification System}
        estimated sucess rate and reasons
        images of system in action
       
 		%\includegraphics[width=.44\linewidth]{../prob1.png}  
 		\hspace{.7cm}
        %\includegraphics[width=.44\linewidth]{../prob2.png}  
        
        
        %\includegraphics[width=.44\linewidth]{../deltaProb1.png}
        \hspace{.7cm}
        %\includegraphics[width=.44\linewidth]{../deltaProb2.png}  
        \end{block}
        
       
      \end{column}
    \end{columns}
     \begin{block}{Conclusion}

        
        \end{block}


  \end{frame}
\end{document}


