\documentclass[final]{beamer}
% beamer 3.10: do NOT use option hyperref={pdfpagelabels=false} !
% \documentclass[final,hyperref={pdfpagelabels=false}]{beamer} 
% beamer 3.07: get rid of beamer warnings

\mode<presentation> {  
%% check http://www-i6.informatik.rwth-aachen.de/~dreuw/latexbeamerposter.php for examples
  \usetheme{Durham} %% This points to the theme cooked up by the final year tutor
}


\usepackage[english]{babel} 
\usepackage[latin1]{inputenc}
\usepackage{amsmath,amsthm, amssymb, latexsym}

  \usefonttheme[onlymath]{serif}
  \boldmath
  \usepackage[orientation=portrait,size=a0,scale=1.4,debug]{beamerposter}                       

  % e.g. for DIN-A0 poster
  % \usepackage[orientation=portrait,size=a1,scale=1.4,grid,debug]{beamerposter}
  % e.g. for DIN-A1 poster, with optional grid and debug output
  % \usepackage[size=custom,width=200,height=120,scale=2,debug]{beamerposter} % e.g. for custom size poster
  % \usepackage[orientation=portrait,size=a0,scale=1.0,printer=rwth-glossy-uv.df]{beamerposter}
  % e.g. for DIN-A0 poster with rwth-glossy-uv printer check ...
  %

  \title[Thermal Re-ID]{Camera-to-Camera Tracking for Person Re-identification within Thermal Imagery}
  \author[G Ingram]{Thomas Robson - Supervised by Dr Toby Breckon}
  \institute[Durham]{School of Computer Science, Durham University}

  \begin{document}
  \begin{frame}{} 

  \vfill
  \begin{block}{Introduction}
          A fundamental task for a distributed multi-camera surveillance system is Person Re-Identification, or to associate people across different camera views. This has been well researched in colour, but there has been very little research done on solving this problem in thermal, making our work state of the art. 

The intent of this project is answer the question "Which features of a human target are appropriate to facilitate Re-Identification in thermal video?" and to develop a functioning Re-Identification system. The challenges associated with this are due to the increased complexity of thermal features compared to colour features.
        \end{block}
        
    \begin{columns}[t]
      \begin{column}{.49\linewidth}
        
        \begin{block}{Person Detection and Tracking}
        	The process of Person detection and Tracking employed in this Project can be broken down into multiple stages. 
        	\begin{itemize}
        	\item Background Subtraction. This is done using the Mixture of Gaussians (MOG) [1] method, allowing the system to learn a background model and comparing each new frame to this. 

        	\item Person Identification. This is done by performing contour dectection on a foreground target and using the Histogram of Oriented Gradients (HOG) [2] to determine whether this target is a person.
        	
        	\item TLD Tracker.  This breaks down the person-tracking task into tracking (following the object between frames), learning (estimating the errors made by the detector and updating it) and detection (correcting the tracker if necessary based on previous observations). 
          \end{itemize}
        \end{block}
       


        \begin{block}{Network Architecture}
        Our network architecture is a Deep Siamese CNN. This means that the network is trained on pairs of images, and determines whether these images show the same person or not. Our CNN consists of convolutional layers, pooling layers and fully connected layers. 
        % more explanation here if room
        \includegraphics[width=.95\linewidth]{../images/architecture.jpg}
        \end{block}
	
        \begin{block}{Network Results 1}
		% maybe loss and acc graphs, but probs not
		
		%tables for conf mat and class rep

        \end{block}
        
        \begin{block}{References}
		\scriptsize
		
[1] Zoran Zivkovic. Improved Adaptive Gaussian Mixture Model for Background Subtraction. Proceedings of the 17th International Conference on Pattern Recognition, 2004.
    
[2] Navneet Dalal and Bill Triggs. Histograms of Oriented Gradients for Human Detection. In IEEE Computer Society Conference on Computer Vision and Pattern Recognition, 2005.
		
    \end{block}
    
    
	 \end{column}
	 \begin{column}{.49\linewidth}
 		\begin{block}{Network Results 2}
 		%euclidean distance plots
 		
 		\includegraphics[width=.32\linewidth]{../images/trainResults.png}  
 		\hspace{.001cm}
 		\includegraphics[width=.32\linewidth]{../images/validationResults.png}
 		\hspace{.001cm}
        \includegraphics[width=.32\linewidth]{../images/evalResults.png}
        \end{block}
        
        \begin{block}{The Re-Identification System}
        estimated sucess rate and reasons
        images of system in action
       
 		%\includegraphics[width=.44\linewidth]{../prob1.png}  
 		\hspace{.7cm}
        %\includegraphics[width=.44\linewidth]{../prob2.png}  
        
        
        %\includegraphics[width=.44\linewidth]{../deltaProb1.png}
        \hspace{.7cm}
        %\includegraphics[width=.44\linewidth]{../deltaProb2.png}  
        \end{block}
        
       	\begin{block}{Conclusion}

        
        \end{block}

      \end{column}
    \end{columns}
     

  \end{frame}
\end{document}


